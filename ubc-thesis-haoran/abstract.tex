%% The following is a directive for TeXShop to indicate the main file
%%!TEX root = diss.tex

\chapter{Abstract}

Metadata in a table help users understand the data contents in the table, in particular, metadata tags can describe the contents in the table and allow a user to easily browse, search, and filter data. However, due to heterogeneity and incompleteness in a table and its metadata, metadata is less useful for the user. Given one table, it is difficult to find all other tables related to the given table by only examining the tags, because the user is typically looking for overlap of tags between two or more tables and there are no such overlaps in the heterogeneous metadata.

We use Open Data tables as a case study, and develop strategies to augment the tags in table metadata to increase the tag overlaps between metadata of different tables. As an initialization step, we perform semantic enrichment of words in attributes of table schema and in tags, and perform schema matching between attributes and tags of a table to create semantic labeling, where an attribute is labeled with zero or more tags. We then provide one table as base table, and we search for tables using the semantic labeling we created to quickly find related tables. We integrated the table searching step and a schema matching step into an iterative framework, which incrementally add additional tags to a table’s metadata for all the tables related to the base table. The additional tags added to the metadata are discovered by semantics overlap during the schema matching step in the iterative framework, based on a composite score with evidence from multiple pairwise value comparison criteria.

We evaluate two approaches we implemented using a gold standard we created, and compare the accuracy of the augmented tags and the runtime with two baseline approaches. We show that the
tags augmented has a relatively high accuracy and the runtime of our iterative approach is reasonable. We argue that an approach that creates approximate matching in a pay-as-you-go fashion has a good precision and recall, and is the more realistic option in a real-world scenario.
