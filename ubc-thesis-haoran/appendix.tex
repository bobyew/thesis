\chapter{Evaluation data sample}

\section{Test Set}
\label{sec:TestSet}

The example below shows an input test set for evaluating the accuracy of a metadata augmentation algorithm. Each test is configured by a repository size k and a specific mix proportion (e.g. 5+0), as well as a base table. The first table in a test is the base table (e.g. “parks” for the “5+0” test). Please refer to Surrey Open Data for the data and the metadata in each table.

\lstset{
    string=[s]{"}{"},
    stringstyle=\color{blue},
    comment=[l]{:},
    commentstyle=\color{black},
}
\begin{lstlisting}
{
  "5+0": [
    "parks",
    "park specimen trees",
    "park screen trees",
    "park outdoor recreation facilities",
    "park structures"
  ],
  "3+2": [
    "parks",
    "park outdoor recreation facilities",
    "park sports fields",
    "water assemblies",
    "road row requirements downtown"
  ],
  "1+4": [
    "parks",
    "water utility facilities",
    "sanitary lift stations",
    "drainage dyke infrastructure",
    "water meters"
  ],
  "10+0": [
    "parks",
    "park paths and trails",
    "park natural areas",
    "park specimen trees",
    "park unimproved parkland",
    "park outdoor recreation facilities",
    "park sports fields",
    "park structures",
    "trails and paths",
    "walking routes"
  ],
  "5+5": [
    "parks",
    "park paths and trails",
    "park natural areas",
    "park specimen trees",
    "park unimproved parkland",
    "heritage sites",
    "aquatic hubs",
    "water fittings",
    "road row requirements downtown",
    "water valves"
  ],
  "1+9": [
    "parks",
    "drainage dyke infrastructure",
    "sanitary lift stations",
    "water utility facilities",
    "water valves",
    "water meters",
    "sanitary nodes",
    "heritage routes",
    "water assemblies",
    "aquatic hubs"
  ]
}
\end{lstlisting}

\section{Gold Standard}
\label{sec:Gold Standard}

The example below shows all the related tables in a gold standard for a test set. If a test repository contains many unrelated tables, these tables will not be in the gold standard (nor their tags).
\lstset{
    string=[s]{"}{"},
    stringstyle=\color{blue},
    comment=[l]{:},
    commentstyle=\color{black},
}
\begin{lstlisting}
{
  "5+0": [
    "parks",
    "park specimen trees",
    "park screen trees",
    "park outdoor recreation facilities",
    "park structures"
  ],
  "3+2": [
    "parks",
    "park outdoor recreation facilities",
    "park sports fields"
  ],
  "1+4": [
    "parks"
  ],
  "10+0": [
    "parks",
    "park paths and trails",
    "park natural areas",
    "park specimen trees",
    "park unimproved parkland",
    "park outdoor recreation facilities",
    "park sports fields",
    "park structures",
    "trails and paths",
    "walking routes"
  ],
  "5+5": [
    "parks",
    "park paths and trails",
    "park natural areas",
    "park specimen trees",
    "park unimproved parkland"
  ],
  "1+9": [
    "parks"
  ]
}
\end{lstlisting}

Each of the related tables in the gold standard contains augmented set of tags. We give the set of tags for one table below.

\begin{description}
\item[Table name:]Park specimen trees
\item[Original tags:][trees]
\item[Augmented tags:]Augmented tags: [status, donation, park, parkland, location, road, residential, facility, green, commercial, environmental, service, trees, parks, facilities, classification]
\end{description}
\endinput