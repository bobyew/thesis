\newglossaryentry{data instance}{name=data instance,description={data has many shapes, sizes and formats; each data instance contains one table and its metadata}}
\newglossaryentry{open data}{name=open data,description={a repository that contains a collection of data instances created and published by many individual authors, and can be accessed by the general public without paying for the data}}
\newglossaryentry{metadata}{name=metadata,description={structured information about a resource of any media type or format; it describes the data}}
\newglossaryentry{tag}{name=tag,description={one type of metadata used in open data, it comes in the form of a list of one or more text items}}
\newglossaryentry{topic}{name=topic,description={a tag that contains semantics such as a dictionary definition}}
\newglossaryentry{table}{name=table,description={a form of data instance, it is a two-dimensional matrix of values, where each column has a header}}
\newglossaryentry{header}{name=header,description={name of a column in a table}}
\newglossaryentry{schema}{name=schema,description={metadata containing a list of mappings, where a header is mapped to a column}}
\newglossaryentry{attribute}{name=attribute,description={another name for a column in a table}}
\newglossaryentry{constraint}{name=constraint,description={restrictions, for example, on the type of data stored in a column}}
\newglossaryentry{heterogeneity}{name=heterogeneity,description={items that resemble the same information are represented differently across different data instances}}
\newglossaryentry{semantic heterogeneity}{name=semantic heterogeneity,description={the same data may have different meanings across different data instances (and vice versa)}}
\newglossaryentry{normalization}{name=normalization,description={ensure each data instance contains all the tags describing the same information}}
\newglossaryentry{metadata augmentation}{name=metadata augmentation,description={an approach to look for existing tags in the related tables; and for all tables lacking these tags, we add the tags to these tables}}
\newglossaryentry{semantic labeling}{name=semantic labeling,description={matching a set of attributes to a set of tags}}
\newglossaryentry{pay-as-you-go}{name=pay-as-you-go,description={an as-needed approach where only the related tables are augmented with tags; only tags that relates to the base table are augmented}}
\newglossaryentry{base table}{name=base table,description={a table the user is familiar with; its data and metadata}}
\newglossaryentry{word sense}{name=word sense,description={the meaning of a word, defined in a dictionary}}
\newglossaryentry{semantic enrichment}{name=semantic enrichment,description={the step to attach definitions to words}}
\newglossaryentry{word sense disambiguation}{name=word sense disambiguation,description={the process of finding the correct interpretation of a word}}
\newglossaryentry{schema matching}{name=schema matching,description={an algorithm to compare two schemata, where attributes in one schema are compared to attributes of the second schema to find a match}}
\newglossaryentry{matching criterion}{name=matching criterion,description={a function that determines the similarity between two attributes}}
\newglossaryentry{correspondence}{name=correspondence,description={a triple from the match in the form (a,b,c), where a, b are attributes from the two schemata, and c is a score according to some matching criteria}}
\newglossaryentry{table searching}{name=table searching,description={an algorithm that searches the tables to find how the tables are related; it compares the tables in a pairwise manner}}
\newglossaryentry{ontology}{name=ontology,description={a collection of related concepts; each concept has a number of properties, and each concept is related to other concepts}}
\newglossaryentry{knowledge base}{name=knowledge base,description={a database storing a collection of facts that can be understood and processed by humans or machines}}
\newglossaryentry{data integration}{name=data integration,description={the process of combining data instances by creating data mappings to capture the relatedness of attributes of the schemata}}
\newglossaryentry{mediated schema}{name=mediated schema,description={a schema with mediated attributes, where each schema has a source description (i.e., a mapping) in terms of the mediated schema}}