%% The following is a directive for TeXShop to indicate the main file
%%!TEX root = diss.tex

\chapter{Glossary}

This glossary uses the handy \latexpackage{acroynym} package to automatically
maintain the glossary.  It uses the package's \texttt{printonlyused}
option to include only those acronyms explicitly referenced in the
\LaTeX\ source.  To change how the acronyms are rendered, change the
\verb+\acsfont+ definition in \verb+diss.tex+.

% use \acrodef to define an acronym, but no listing
\acrodef{UI}{user interface}
\acrodef{UBC}{University of British Columbia}

% The acronym environment will typeset only those acronyms that were
% *actually used* in the course of the document
\begin{acronym}[ANOVA]
\acro{ANOVA}[ANOVA]{Analysis of Variance\acroextra{, a set of
  statistical techniques to identify sources of variability between groups}}
\acro{API}{application programming interface}
\acro{CTAN}{\acroextra{The }Common \TeX\ Archive Network}
\acro{DOI}{Document Object Identifier\acroextra{ (see
    \url{http://doi.org})}}
\acro{GPS}[GPS]{Graduate and Postdoctoral Studies}
\acro{PDF}{Portable Document Format}
\acro{RCS}[RCS]{Revision control system\acroextra{, a software
    tool for tracking changes to a set of files}}
\acro{TLX}[TLX]{Task Load Index\acroextra{, an instrument for gauging
  the subjective mental workload experienced by a human in performing
  a task}}
\acro{UML}{Unified Modelling Language\acroextra{, a visual language
    for modelling the structure of software artefacts}}
\acro{URL}{Unique Resource Locator\acroextra{, used to describe a
    means for obtaining some resource on the world wide web}}
\acro{W3C}[W3C]{\acroextra{the }World Wide Web Consortium\acroextra{,
    the standards body for web technologies}}
\acro{XML}{Extensible Markup Language}

\acro{data instance}{data in many shapes and formats, each data instance contains one table and metadata describing the data instance}
\acro{open data}{repository that contains a collection of data instances created and published by many individual authors, can be accessed by the general public without paying for the data}
\acro{metadata}{structured information about an information resource of any media type or format}

\acro{tag}{one type of metadata used in open data, it comes in the form of a list of one or more text items}
\acro{topic}{a tag that contains semantics such as a dictionary definition}
\acro{table}{a form of data instance, it is a two-dimensional matrix of values, where each column has a header}
\acro{header}{name of a column in a table}
\acro{schema}{metadata containing a list of mappings, where a header is mapped with a column}
\acro{attribute}{header within a schema}
\acro{constraint}{restrictions, for example, on the type of data stored in a column}
\acro{heterogeneity}{items that resemble the same information is represented differently across different data instances}
\acro{semantic heterogeneity}{the same data may have different meanings across different data instances (and vice versa)}
\acro{normalization}{ensure each data instance contains all the tags describing the same information}
\acro{metadata augmentation}{an approach to look for existing tags in the related tables, and for all tables lacking these tags, we add the tags to these tables}
\acro{semantic labeling}{matching between a set of attributes and a set of tags}
\acro{pay-as-you-go}{an approach where only the related tables are augmented with tags, and only tags that relates to the base table are augmented}
\acro{base table}{a table the user is familiar with its data and metadata}
\acro{word sense}{the meaning of a word, defined in a dictionary}
\acro{semantic enrichment}{the step to attach definitions to words}
\acro{word sense disambiguation}{the process of finding the correct interpretation of a word}
\acro{schema matching}{an algorithm that compare between a pair schemata, where attributes in one schema are compared to attributes of the second schema to find a matching}
\acro{matching criterion}{a function that determines the similarity between two attributes}
\acro{correspondence}{a pair from the matching in the form of (a,b,c), where a, b are attributes from the two schemata, and c is a score according to some matching criteria}
\acro{table searching}{an algorithm that searches the tables to find how tables are related, the query is compared with every other table in a pairwise manner}
\acro{ontology}{a collection of related concepts, each concept has a number of properties, and each concept is related to other concepts}
\acro{knowledge base}{a database storing a collection of facts that can be understood and processed by humans or machines}
\acro{data integration}{combining data instances by creating data mappings to capture the relatedness of attributes of the schemata}
\acro{mediated schema}{a schema with mediated attributes, where each schema has a source description (i.e., a mapping) in terms of the mediated schema}

\end{acronym}


% You can also use \newacro{}{} to only define acronyms
% but without explictly creating a glossary
% 
% \newacro{ANOVA}[ANOVA]{Analysis of Variance\acroextra{, a set of
%   statistical techniques to identify sources of variability between groups.}}
% \newacro{API}[API]{application programming interface}
% \newacro{GOMS}[GOMS]{Goals, Operators, Methods, and Selection\acroextra{,
%   a framework for usability analysis.}}
% \newacro{TLX}[TLX]{Task Load Index\acroextra{, an instrument for gauging
%   the subjective mental workload experienced by a human in performing
%   a task.}}
% \newacro{UI}[UI]{user interface}
% \newacro{UML}[UML]{Unified Modelling Language}
% \newacro{W3C}[W3C]{World Wide Web Consortium}
% \newacro{XML}[XML]{Extensible Markup Language}
\endinput